% Method of the month: Cure survival models for medicine persistence
% (c) 2022 Malcolm Gillies <malcolm.gillies@unsw.edu.au>
% https://github.com/mbg-unsw/curemodel
%
% This work is licensed under a
% Creative Commons Attribution-NonCommercial-ShareAlike 4.0
% International Licence
\documentclass[aspectratio=169,12pt]{beamer} % XXXX fix AR here
\usepackage[latin1]{inputenc}
\usepackage[T1]{fontenc}
\usepackage{textcomp}
\usefonttheme{serif} % need this with Charter font
\usetheme{Berlin}  % using default now
\usecolortheme{beaver}  % using default now
\usepackage[libertine]{libertine} % not using osf (old-style figures)
\usepackage[scale=0.9]{tgheros} % scale to match libertine
\usepackage[varqu,varl]{inconsolata}
\usepackage[libertine]{newtxmath}
\usepackage{amsmath}
\usepackage{graphicx}
\usepackage{tikz}
\usetikzlibrary{shadows}
\usepackage{tikzpagenodes}
\usepackage[round]{natbib}
\usepackage{gitinfo2}
\usepackage{listings}

\renewcommand{\gitMark}{\color{gray}\texttt{\tiny\gitBranch\,@\,\gitAbbrevHash\,\gitAuthorDate}}

\setbeamertemplate{navigation symbols}{} % remove navigation symbols
\setbeamercolor*{item}{fg=darkred}

\title{``Method of the month'': Mixture cure survival models for medicine persistence}
\author{Malcolm Gillies}
\date{19 May 2022}
\usebackgroundtemplate{%
\begin{tikzpicture}[remember picture,overlay]
    \node[anchor=south west,scale=1,rotate=90] at ([shift={(0cm,0cm)}]current page marginpar area.south east) {\gitMark};
\end{tikzpicture}%
}

\newif\ifsidebartheme
\sidebarthemefalse

\newdimen\contentheight
\newdimen\contentwidth
\newdimen\contentleft
\newdimen\contentbottom
\makeatletter
\newcommand*{\calculatespace}{%
    \contentheight=\paperheight%
    \ifx\beamer@frametitle\@empty%
        \setbox\@tempboxa=\box\voidb@x%
      \else%
        \setbox\@tempboxa=\vbox{%
          \vbox{}%
          {\parskip0pt\usebeamertemplate***{frametitle}}%
        }%
        \ifsidebartheme%
          \advance\contentheight by-1em%
        \fi%
      \fi%
    \advance\contentheight by-\ht\@tempboxa%
    \advance\contentheight by-\dp\@tempboxa%
    \advance\contentheight by-\beamer@frametopskip%
    \ifbeamer@plainframe%
    \contentbottom=0pt%
    \else%
    \advance\contentheight by-\headheight%
    \advance\contentheight by\headdp%
    \advance\contentheight by-\footheight%
    \advance\contentheight by4pt%
    \contentbottom=\footheight%
    \advance\contentbottom by-4pt%
    \fi%
    \contentwidth=\paperwidth%
    \ifbeamer@plainframe%
    \contentleft=0pt%
    \else%
    \advance\contentwidth by-\beamer@rightsidebar%
    \advance\contentwidth by-\beamer@leftsidebar\relax%
    \contentleft=\beamer@leftsidebar%
    \fi%
}
\makeatother

\begin{document}

{
%\usebackgroundtemplate{}
\begin{frame}
\titlepage
\end{frame}
}

\begin{frame}{Today's paper}
\calculatespace%
\begin{columns}
\begin{column}{0.20\contentwidth}
\begin{tikzpicture}
  \node[drop shadow={shadow xshift=.8ex,shadow yshift=-.8ex},fill=white,draw] at (0,0) {\includegraphics[width=\textwidth]{ref/cai_sm.jpg}};
\end{tikzpicture}
\end{column}
\begin{column}{0.70\contentwidth}
	\begin{itemize}
		\item C.~Cai et al. \textbf{Applying mixture cure survival modeling to medication persistence analysis} \emph{Pharmacoepidemiol Drug Saf}, 2022;1--8. doi:10.1002/pds5441.
\nocite{cai_applying_2022}
\nocite{cai_smcure_2012}
\nocite{othus_cure_2012}
\nocite{flexsurvcure_2020}
	\end{itemize}
\end{column}
\end{columns}
\end{frame}

\begin{frame}{Recap of survival analysis}
	\begin{itemize}
		\item Time-to-event analysis is fundamental to cohort studies
		\item Unbiased estimates require proper handling of censoring
		\item Kaplan--Meier analysis estimates empirical survival curve
		\item Cox regression allows semiparametric estimation (PH assumption)
	\end{itemize}
\end{frame}

\begin{frame}{Cure survival models}
	\begin{itemize}
		\item Simple survival models consider all cohort members as susceptible
		\begin{itemize}
			\item Good assumption for short follow-up
		\end{itemize}
	\item But if there is a long plateau at the tail of the survival curve
		\begin{itemize}
			\item Evidence for a ``cured'' fraction
		\end{itemize}
	\item \emph{By definition, 100\% of cured cohort members will be censored}
	\end{itemize}
\end{frame}

\begin{frame}{Mixture cure models 1}
\calculatespace%
\begin{columns}
\begin{column}{0.5\contentwidth}
  \centering
  \includegraphics[height=0.7\textheight]{ref/surv1.pdf}

  A bit of this...
\end{column}
\begin{column}{0.5\contentwidth}
  \centering
  \includegraphics[height=0.7\textheight]{ref/surv2.pdf}

  and a little bit of that
\end{column}
\end{columns}
\end{frame}

\begin{frame}{Mixture cure models 2}
\calculatespace%
\begin{columns}
\begin{column}{0.5\contentwidth}
  \centering
  \includegraphics[height=0.7\textheight]{ref/surv3.pdf}
\end{column}
\begin{column}{0.5\contentwidth}
	$S_{pop}(t|X,Z)=\underbrace{\pi(Z)S_u(t|X)}_\text{uncured}+
	\underbrace{1-\pi(Z)}_\text{cured}$
\end{column}
\end{columns}
\end{frame}

%\usebackgroundtemplate{\includegraphics[width=\paperwidth]{ref/newey-west-cites.PNG}}
%\begin{frame}[plain,b]
%\begin{flushright}\texttt{https://app.dimensions.ai/}\end{flushright}
%\end{frame}
%\usebackgroundtemplate{}

\begin{frame}{What's this got to do with medicine persistence?}
	\begin{itemize}
		\item Time-to-event analysis is the obvious way to measure persistence, with discontinuation as the event
		\item Empirically, there may be distinct short- and long-term persistence patterns
		\item In chronic disease, consider the long-term persistent as the ``cured'' fraction
		\item If there is cured fraction, PH assumption is violated
	\end{itemize}
\end{frame}

\begin{frame}{Cai et al example 1}
	\begin{itemize}
		\item South Carolina health Plan and Medicaid
		\item Prescription claims 2008 to 2019
		\item Statin new users (12-month washout)
		\item Participants who died excluded
		\item Age group, gender, comorbidity and insurance covariates
		\item Results shown for 180-day permissible gap
	\end{itemize}
\end{frame}

\begin{frame}{Cai et al example 2}
  \centering
  \includegraphics[height=0.7\textheight]{ref/cai_fig2.pdf}
  \flushright{\tiny{\citet{cai_applying_2022}}}
\end{frame}

\begin{frame}{Cai et al example 3}
\calculatespace%
\begin{columns}
\begin{column}{0.5\contentwidth}
  \centering
  \includegraphics[width=0.9\textwidth]{ref/cai_tab2.pdf}
\end{column}
\begin{column}{0.5\contentwidth}
  \centering
  \includegraphics[height=0.6\textheight]{ref/cai_tab3.pdf}
\end{column}
\end{columns}
\flushright{\tiny{\citet{cai_applying_2022}}}
\end{frame}

\begin{frame}{Statistical inference}
    \begin{itemize}
        \item Test if the cured fraction > 0
	\item Test if the follow up is long enough
	\item \emph{n.b. in the Cai et al (2022) paper this is done using parametric models}
    \end{itemize}
\end{frame}

\lstset{
basicstyle=\tiny\ttfamily,basewidth=0.5em
}
\begin{frame}[fragile=singleslide]{Worked (trivial) example in R}
	\begin{lstlisting}
> head(disp)
    d cens
1 129    0
2  28    0
3  47    0
4 223    0
5 129    0
6  21    0
> flexsurvcure(Surv(d, 1-cens)~1, data=disp, dist="exp", mixture=T)
Call:
flexsurvcure(formula = Surv(d, 1 - cens) ~ 1, data = disp, dist = "exp",
    mixture = T)

Estimates:
       est       L95%      U95%      se
theta  0.095714  0.073067  0.124438        NA
rate   0.010287  0.009380  0.011282  0.000484

N = 1000,  Events: 863,  Censored: 137
Total time at risk: 112600
Log-likelihood = -5044.493, df = 2
AIC = 10092.99
\end{lstlisting}
\end{frame}

\begin{frame}{The wild west of survival models...}
	\begin{itemize}
		\item There's more to life than just Cox models:
			\begin{itemize}
				\item Parametric survival
				\item Accelerated failure time
				\item Shared/conditional frailty
				\item ...
			\end{itemize}
	\end{itemize}
\end{frame}

\begin{frame}{Summary}
	\begin{itemize}
		\item If short-term and long-term persistence are distinct, try a cure model
		\item Estimating cure models can be tricky
		\item Statistical tests can clarify which model to use
		\item Still many parameters to twiddle e.g. permissable gap
		\item Watch this space, there is plenty of room for new ideas
	\end{itemize}
\end{frame}

\begin{frame}{Thanks}
    \begin{itemize}
        \item Andrea
    \end{itemize}
\end{frame}

\begin{frame}{References}
%        \bibliographystyle{apalike}
%        \bibliographystyle{abbrv}
        \tiny\bibliography{cure.bib}
        \bibliographystyle{abbrvnat}
\end{frame}

\end{document}
